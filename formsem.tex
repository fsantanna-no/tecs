\documentclass[11pt,a4paper,oneside]{amsart}
\title{Notes on the semantics of Céu}
\renewcommand{\datename}{}
\date{Guilherme F.~Lima, \today}
%---------------------------------------------------------------------------
% TODO:
% 1. Adicionar as instruções @\bot (erro) e @\top (terminação).
% 2. Tornar a instrução 'fin p' bloqueante.
% 3. Tentar tirar o número n da relação \step.
% 4. Adicionar sequência de saída (emit p. eventos externos) à configuração.
% 5. Completar as provas: determinismo, terminação, ordenação das
%    finalizações.
% 6. Definir relação de equivalência entre programas.
%
% FUTURO:
% - Comparar as semânticas construtivas de Esterel.
% - Colocar comportamentos (behaviors, instrução spawn) na semântica.
%---------------------------------------------------------------------------
\usepackage[protrusion=true]{microtype}
\usepackage[utf8]{inputenc}
\usepackage[T1]{fontenc}
\usepackage{aliascnt}
\usepackage{amsmath}
\usepackage{amsthm}
\usepackage{hyperref}
\hypersetup{colorlinks=true,allcolors=black}
\usepackage{mathtools}
\usepackage[varg]{txfonts}
\usepackage{xcolor}
\usepackage[nameinlink]{cleveref}

% To-do.
\def\FIXME#1{{\color{red}{\textbf{#1}}}}
\newenvironment{TODO}
  {\bigskip\bgroup\color{red}\textbf{TODO}.~}
  {\egroup\bigskip}

% Theorems.
\theoremstyle{plain}
\newtheorem{thm}{Theorem}
\Crefname{thm}{Theorem}{Theorems}

\theoremstyle{definition}
\newaliascnt{defn}{thm}
\newtheorem{defn}[defn]{Definition}
\aliascntresetthe{defn}
\Crefname{defn}{Definition}{Definitions}

\theoremstyle{remark}
\newtheorem*{remark}{Remark}
\Crefname{remark}{Remark}{Remarks}

% Bussproofs.
\usepackage{bussproofs}
\def\labelSpacing{0em}
\def\ScoreOverhang{0em}
\def\defaultHypSeparation{\quad}
\EnableBpAbbreviations

% Auxiliary.
\def\R#1{\hyperref[Rule:#1]{$R_\text{#1}$}}
\def\Rtag#1{\tag{$R_\text{#1}$}}

% Symbols.
\def\Ceu{C\'eu}
\let\nil=\varepsilon
\def\<#1>{\langle#1\rangle}
\def\|#1|{\left|#1\right|}
\def\val{\mathop{\mathit{val}}}
\let\clear=\kappa
\makeatletter
\def\@raise#1#2#3{
  \setbox0=\hbox{#1}%
  \mathbin{%
    \hbox to\wd0{%
      \rlap{\box0}\hfill\raise#2\hbox{$\scriptstyle{#3}$}\hfill
    }%
  }%
}
\def\step#1{\@raise{$\to$}{1.1ex}{#1}}
\def\stepx#1{\@raise{$\step{#1}$}{-.75ex}{\ast}}
\def\eval#1{\@raise{$\Rightarrow$}{1.1ex}{#1}}
\def\quadtext#1{\quad\text{#1}\quad}
\def\Case#1.{[Case~#1]\enskip}
\makeatother

% Abstract syntax.
\makeatletter
\def\@ceuop#1{\mathop{\texttt{#1}}}%
\def\@ceubin#1{\mathbin{\texttt{#1}}}%
\def\ceu#1{%
  \bgroup
  \def\Skip{\@ceuop{skip}}%
  \def\Mem{\@ceuop{mem}}%
  \def\Await{\@ceuop{await}}%
  \def\Emit{\@ceuop{emit}}%
  \def\Break{\@ceuop{break}}%
  \def\Ifelse##1##2##3{%
    \@ceuop{if}%
    \@ceuop{mem}(##1)%
    \@ceuop{then}{##2}%
    \@ceuop{else}{##3}}%
  \def\Loop{\@ceuop{loop}}%
  \def\And{\@ceubin{and}}%
  \def\Or{\@ceubin{or}}%
  \def\Fin{\@ceuop{fin}}%
  \def\Awaiting{\@ceuop{@awaiting}}%
  \def\Emitting{\@ceuop{@emitting}}%
  \def\Atloop{\@ceuop{@loop}}%
  \ensuremath{#1}\ignorespaces
  \egroup
}
\makeatother

\begin{document}
\maketitle


\section{Abstract syntax}
\label{Section:syntax}

The \emph{abstract syntax} of \Ceu\ programs is given by the following
grammar:
%%
\bgroup
\vskip\abovedisplayskip
\noindent
\hfil\hbox{%
  \vtop{%
    \tabskip0pt
    \offinterlineskip
    \halign{\strut\hfil$#$&$\;#$\hfil&\qquad#\hfil\cr
      p\in{P}\Coloneqq
          & \ceu{\Mem(v)}         & read or write variable~$v$\cr
      \mid& \ceu{\Await(e)}       & await event~$e$\cr
      \mid& \ceu{\Emit(e)}        & emit event~$e$\cr
      \mid& \ceu{\Break}          & break innermost loop\cr
      \mid& \ceu{\Ifelse{v}{p_1}{p_2}}  & conditional\cr
      \mid& \ceu{p_1;p_2}         & sequence\cr
      \mid& \ceu{\Loop p_1}       & repetition\cr
      \mid& \ceu{p_1\And p_2}     & par/and\cr
      \mid& \ceu{p_1\Or p_2}      & par/or\cr
      \mid& \ceu{\Fin p_1}        & finalization\cr
      \mid& \ceu{\Awaiting(e,n)}  & awaiting~$e$ since reaction~$n$\cr
      \mid& \ceu{\Emitting(n)}    & emitting~$e$ on stack level~$n$\cr
      \mid& \ceu{p_1\Atloop p_2}  & unwinded loop,\cr
    }%
  }%
}\hfil%
\vskip\belowdisplayskip
\egroup
%%
\noindent where~$n\in{N}$ is an integer, $v\in{V}$~is a memory location
(variable) identifier, $e\in{E}$~is an event identifier, and~$p$, $p_1$,
$p_2\in{P}$ are programs.  We use the abbreviation~``\ceu\Skip'' (do
nothing) for the vacuous memory access~``\ceu{\Mem(\nil)}'', where~$\nil$ is
the empty string.  We thus assume that~$\nil\in{V}$.


\section{Reaction step}
\label{Section:step}

For our purposes, the \emph{state} of a \Ceu\ program within a reaction is
represented by a stack of event identifiers~$e_1e_2\dots{}e_n\in{}E^*$,
where the leftmost element is the top-of-stack.  A \emph{configuration} is a
triple~$\<p,\alpha,n>\in{P\times{E^*}\times{N}}=\Delta$ that represents the
situation of program~$p$ waiting to be evaluated in state~$\alpha$ and
reaction~$n$.  Given an initial configuration, each program reaction is
computed by successive applications of the reaction-step
relation~$\mathord{\to}\subseteq{\Delta\times\Delta}$ such
that~$\<p,\alpha,n>\step{}\<p',\alpha',n>$ iff: an execution step of
program~$p$ in state~$\alpha$ and reaction~$n$ evaluates to a modified
program~$p'$ and a modified state~$\alpha'$ in the same reaction.  Since
relation~$\step{}$ is defined in a way such that it can only relate
configurations with the same~$n$, we often
write~$\<p,\alpha>\step{n}\<p',\alpha'>$
for~$\<p,\alpha,n>\step{}\<p',\alpha',n>$.\footnote{The reaction number~$n$
  is used only to prevent~\ceu\Await\ instructions from awaking in the same
  reaction they are reached.}

The reaction-step relation~$\step{\null}$ is defined inductively by the
rules presented in
%%
\Cref{%
Section:step:await-and-emit,%
Section:step:conditionals,%
Section:step:sequences-and-loops,%
Section:step:parand-and-paror,%
}.


\subsection{Await and emit}
\label{Section:step:await-and-emit}

For all~$n$, $n'\in{N}$, $e\in{E}$, and~$\alpha\in{E^*}$:
\begin{alignat*}{2}
\Rtag{await}
\<\ceu{\Await(e)},\alpha>
&\step{n}\<\ceu{\Awaiting(e,n')},\alpha>
&&\qquad\text{with~$n'=n+1$}\\
%%
\Rtag{awake}
\<\ceu{\Awaiting(e,n')},e\alpha>
&\step{n}\<\ceu\Skip,e\alpha>
&&\qquad\text{if~$n'\leq{n}$}\\
%%
\Rtag{emit-push}
\<\ceu{\Emit(e)},\alpha>
&\step{n}\<\ceu{\Emitting(n')},e\alpha>
&&\qquad\text{with~$n'=\|\alpha|$}\\
%%
\Rtag{emit-pop}
\<\ceu{\Emitting(n')},\alpha>
&\step{n}\<\ceu\Skip,\alpha>
&&\qquad\text{if~$n'=\|\alpha|$\,.}
\end{alignat*}


\subsection{Conditionals}
\label{Section:step:conditionals}

For all~$n\in{N}$, $v\in{V}$, $\alpha\in{E^*}$, and~$p_1$, $p_2\in{P}$:
\begin{alignat*}{3}
\Rtag{if-true}
\<\ceu{\Ifelse{v}{p_1}{p_2}},\alpha>
&\step{n}\<p_1,\alpha>
&&\qquad\text{if~$\val(v,n)\ne0$}\\
%%
\Rtag{if-false}
\<\ceu{\Ifelse{v}{p_1}{p_2}},\alpha>
&\step{n}\<p_2,\alpha>
&&\qquad\text{if~$\val(v,n)=0$\,,}
\end{alignat*}
%%
where~$\val(v,n)$ denotes the value of variable~$v$ in reaction~$n$.  For
simplicity, and without loss of generality, we deal only with integer
variables, i.e., $\val(v,n)\in{N}$, for all~$v$ and~$n$.  Moreover, we
assume that~$\val(\nil,n)=0$, for all~$n$.


\subsection{Sequences and loops}
\label{Section:step:sequences-and-loops}

For all~$n\in{N}$, $v\in{V}$, $\alpha$, $\alpha'\in{E^*}$, and~$p$, $p_1$,
$p_1'$, $p_2\in{P}$:
\begin{alignat*}{2}
\Rtag{seq-mem}
\<\ceu{\Mem(v);p},\alpha>&\step{n}\<p,\alpha>&&\\
%%
\Rtag{seq-brk}
\<\ceu{\Break};p,\alpha>&\step{n}\<\ceu{\Break},\alpha>&&\\
%%
\Rtag{seq-adv}&\hskip-4.15em
\AXC{$\<p_1,\alpha>\step{n}\<p_1',\alpha'>$}
\UIC{$\<p_1;p_2,\alpha>\step{n}\<p_1';p_2,\alpha'>$}
\DP&&\\
%%
\Rtag{loop-expd}
\<\ceu{\Loop p},\alpha>&\step{n}\<\ceu{p\Atloop p},\alpha>&&\\
%%
\Rtag{loop-mem}
\<\ceu{\Mem(v)\Atloop p},\alpha>&\step{n}\<\ceu{\Loop p},\alpha>&&\\
%%
\Rtag{loop-brk}
\<\ceu{\Break\Atloop p},\alpha>&\step{n}\<\ceu\Skip,\alpha>&&\\
%%
\Rtag{loop-adv}&\hskip-6.7em
\AXC{$\<p_1,\alpha>\step{n}\<p_1',\alpha'>$}
\UIC{$\<\ceu{p_1\Atloop p_2},\alpha>
  \step{n}\<\ceu{p_1'\Atloop p_2},\alpha'>$}
\DP&&
\end{alignat*}


\subsection{Par/and and par/or}
\label{Section:step:parand-and-paror}

For all~$n\in{N}$, $v\in{V}$, $\alpha$, $\alpha'\in{E^*}$, and~$p_1$,
$p_1'$, $p_2$, $p_2'\in{P}$:
\begin{alignat*}{2}
\Rtag{and-mem1}
\<\ceu{\Mem(v)\And p_2},\alpha>&\step{n}\<p_2,\alpha>&&\\
%%
\Rtag{and-mem2}
\<\ceu{p_1\And\Mem(v)},\alpha>&\step{n}\<p_1,\alpha>&&\\
%%
\Rtag{and-brk1}
\<\ceu{\Break{}\And p_2},\alpha>&\step{n}\<\ceu{p_2';\Break},\alpha>
&&\quad\text{with~$p_2'=\clear(p_2)$}\\
%%
\Rtag{and-brk2}
\<\ceu{p_1 \And\Break{}},\alpha>&\step{n}\<\ceu{p_1';\Break},\alpha>
&&\quad\text{with~$p_1'=\clear(p_1)$}\\
%%
\Rtag{and-adv1}&\hskip-5.75em
\AXC{$\<p_1,\alpha>\step{n}\<p_1',\alpha'>$}
\UIC{$\<\ceu{p_1\And p_2},\alpha>\step{n}\<\ceu{p_1'\And p_2},\alpha'>$}
\DP&&\\
%%
\Rtag{and-adv2}&\hskip-5.75em
\AXC{$\<p_2,\alpha>\step{n}\<p_2',\alpha'>$}
\UIC{$\<\ceu{p_1\And p_2},\alpha>\step{n}\<\ceu{p_1\And p_2'},\alpha'>$}
\DP&&\quad\text{if~$\#(p_1,\alpha,n)$\,,}
\end{alignat*}
where the call~$\clear(p)$ (read ``clear~$p$'') concatenates in a sequence
the body~$p'$ of all active finalization instructions~\ceu{\Fin{p'}} in~$p$,
and the predicate~$\#(p,\alpha,n)$ (read ``$p$ is blocked'') is true iff all
execution trails of~$p$ are hanged in instructions that cannot be consumed
in the current reaction step, viz., \ceu\Awaiting\ or \ceu\Emitting.  See
\Cref{Appendix:extra:clear,Appendix:extra:is-blocked} for the precise
definition of these functions.

The evaluation rules for par/or instructions (below) are slightly different.
While in the case par/and both sides must terminate (evaluate
to~\ceu{\Mem(v)} or~\ceu\Break) for the composition to terminate, in the
case of par/or the termination of \emph{any} side causes the whole
composition to terminate.  Note, however, that in both cases, par/and and
par/or, the right-hand side of the instruction can only be evaluated if the
its left-hand side is blocked.

For all~$n\in{N}$, $v\in{V}$, $\alpha$, $\alpha'\in{E^*}$, and~$p_1$,
$p_1'$, $p_2$, $p_2'\in{P}$:
\begin{alignat*}{2}
\Rtag{or-mem1}
\<\ceu{\Mem(v)\Or p_2},\alpha>&\step{n}\<\ceu{p_2'},\alpha>
&&\enspace\text{with~$p_2'=\clear(p_2)$}\\
%%
\Rtag{or-mem2}
\<\ceu{p_1\Or\Mem(v)},\alpha>&\step{n}\<\ceu{p_1'},\alpha>
&&\enspace\text{if~$\#(p_1,\alpha,n)$, with~$p_1'=\clear(p_1)$}\\
%%
\Rtag{or-brk1}
\<\ceu{\Break{}\Or p_2},\alpha>&\step{n}\<\ceu{p_2';\Break},\alpha>
&&\enspace\text{with~$p_2'=\clear(p_2)$}\\
%%
\Rtag{or-brk2}
\<\ceu{p_1\Or\Break},\alpha>&\step{n}\<\ceu{p_1';\Break},\alpha>
&&\enspace\text{if~$\#(p_1,\alpha,n)$, with~$p_1'=\clear(p_1)$}\\
%%
\Rtag{or-adv1}&\hskip-5.2em
\AXC{$\<\ceu{p_1},\alpha>\step{n}\<p_1',\alpha'>$}
\UIC{$\<\ceu{p_1\Or p_2},\alpha>\step{n}\<\ceu{p_1'\Or p_2},\alpha'>$}
\DP&&\\
%%
\Rtag{or-adv2}&\hskip-5.2em
\AXC{$\<\ceu{p_2},\alpha>\step{n}\<p_2',\alpha'>$}
\UIC{$\<\ceu{p_1\Or p_2},\alpha>\step{n}\<\ceu{p_1\Or p_2'},\alpha'>$}
\DP&&\enspace\text{if~$\#(n,\alpha,p_1)$}
\end{alignat*}


\section{Properties of the reaction-step relation}
\label{Section:step-properties}

\begin{thm}[Determinism]
For all~$p$, $p_1$, $p_2\in{P}$, $\alpha$, $\alpha_1$, $\alpha_2\in{E^*}$,
and~$n\in{N}$:
\begin{equation*}
(\<p,\alpha>\step{n}\<p_1,\alpha_1>
\quadtext{and}
\<p,\alpha>\step{n}\<p_2,\alpha_2>)
\quadtext{implies}
\<p_1,\alpha_1>=\<p_2,\alpha_2>\,.
\end{equation*}
\end{thm}
\begin{proof}
Suppose~$\<p,\alpha>\step{n}\<p_1,\alpha_1>$
and~$\<p,\alpha>\step{n}\<p_2,\alpha_2>$, for arbitrary~$p$, $p_1$,
$p_2\in{P}$, $\alpha$, $\alpha_1$, $\alpha_2\in{E^*}$, and~$n\in{N}$.  We
proceed by structural induction on~$p$.  The following ten cases are
possible.  (The cases where~$p=\ceu{\Mem(v)}$, $p=\ceu{\Break}$, and
$p=\ceu{\Fin{p'}}$ are not considered since there are no rules to evaluate
them, i.e., by the theorem hypothesis, $p\ne\ceu{\Mem(v)}$,
$p\ne\ceu\Break$, and~$p\ne\ceu{\Fin{p'}}$.)

\Case{1}. $p=\ceu{\Await(e)}$, for some~$e\in{E}$.  By rule~\R{await},
$p_1=p_2=\ceu{\Awaiting(e,n')}$, with~$n'=n+1$,
and~$\alpha_1=\alpha_2=\alpha$.

\Case{2}. $p=\ceu{\Awaiting(e,n')}$, for some~$e\in{E}$.  By rule~\R{awake},
$p_1=p_2=\ceu\Skip$, with~$n'\leq{n}$, and~$\alpha_1=\alpha_2=\alpha$.

\Case{3}. $p=\ceu{\Emit(e)}$, for some~$e\in{E}$.
\FIXME{\dots}

\Case{4}. $p=\ceu{\Emitting(n')}$, for some~$n'\in{N}$.
\FIXME{\dots}

\Case{5}. $p=\ceu{\Ifelse{v}{p'}{p''}}$, for some~$v\in{V}$ and~$p'$,
$p''\in{P}$.
\FIXME{\dots}

\Case{6}. $p=\ceu{p';p''}$, for some~$p'$, $p''\in{P}$.
\FIXME{\dots}

\Case{7}. $p=\ceu{\Loop p'}$, for some~$p'\in{P}$.
\FIXME{\dots}

\Case{8}. $p=\ceu{p'\Atloop p''}$, for some~$p'$, $p''\in{P}$.
\FIXME{\dots}

\Case{9}. $p=\ceu{p'\And p''}$, for some~$p'$, $p''\in{P}$.
\FIXME{\dots}

\Case{10}. $p=\ceu{p'\Or p''}$, for some~$p'$, $p''\in{P}$.
\FIXME{\dots}

\begin{TODO}
Aqui eu esbarro no primeiro problema.  A indução em~$P$ fica
  complicada porque há programas que não avaliam para nada.  Por exemplo,
  se~$p=\ceu{\Ifelse{v}{p'}{p''}}$ com $p'=\ceu\Break$, então a hipótese
  indutiva não vale~$p'$, mesmo com~$p'\prec{p}$.  Ou seja, eu teria que
  identificar todos os casos em que o~$p'$ da hipótese indutiva não vale.
  Eu vejo duas soluções simples para esse problema.

  A primeira solução é fazer com que todo programa avalie para alguma coisa,
  e.g., definir regras do tipo:
  \begin{align*}
    \<\ceu{\Mem(v)},\alpha>&\step{n}\<\ceu\Skip,\alpha>\\
    \<\ceu\Break,\alpha>&\step{n}\<\ceu\Break,\alpha>\\
    \<\ceu{\Fin{p_1}},\alpha>&\step{n}\<\ceu{\Fin{p_1}},\alpha>\,.\\
  \end{align*}
  Nesse caso, teríamos que ver se as regras acima são suficientes e se elas
  não causam conflitam com as existentes.

  A segunda solução, é deixar em~$P$ apenas os programas que avaliam para
  alguma coisa.  Ou seja, \ceu{\Mem(v)} e \ceu\Break\ não fariam parte
  de~$P$ mas sim de um conjunto auxiliar.  Dessa forma, quando eu fizesse
  uma indução em~$P$ não teria que lidar com esses casos.  Voltado ao
  exemplo, \ceu{\Ifelse{v}{\Break}{p''}} seria tratado como um caso base, já
  que na gramática eu teria algo do tipo:
  \begin{align*}
    p\Coloneqq\ceu{\Ifelse{v}{\Break}{p_2}}\mid\cdots{}
  \end{align*}
  A desvantagem dessa segunda opção é que o número de regras pode aumentar
  significativamente.

  Qualquer que seja a solução, se eu tenho~$\<p,\alpha>$ sempre avaliando
  para alguma coisa, na hora de fazer o fecho (i.e., na definição da
  relação~$\eval{}$) eu não preciso do fecho reflexivo---o transitivo basta.
\end{TODO}
\end{proof}

\begin{thm}[Termination]
For all~$\<p,\alpha>\in\Delta$,
\begin{equation*}
(\exists!\<p',\alpha'>\in\Delta)(\<p,\alpha>\step{n}\<p',\alpha'>\,).
\end{equation*}
\end{thm}
\begin{proof}
By structural induction on~$P$.

\begin{TODO}
Aqui eu caio no memso problema anterior.  Há programas em~$P$ que não
avaliam para nada.
\end{TODO}
\end{proof}


\section{Reaction}
\label{Section:reaction}

\begin{TODO}
Eu tinha definido a seguinte regra:
%%
\begin{quote}
  For all~$n\in{N}$, $e\in{E}$, $\alpha\in{E^*}$, and~$p\in{P}$:
  \label{Rule:pop}
  \begin{equation*}
    \Rtag{pop}\hskip1.7em
    \<\ceu{p},e\alpha>\step{n}\<\ceu{p},\alpha>
    \qquad\qquad\text{if~$\#(p,e\alpha,n)$\,.}
  \end{equation*}
  %%
  This rule is necessary to ensure that pending~\ceu\Emitting\ instructions
  (in lower stack indexes) are eventually resumed by subsequent reaction
  steps.
\end{quote}
Dessa forma, o ``pop'' faria parte da própria definição de passo.  Mas
depois eu percebi que essa regra complica (inviabiliza?) qualquer indução
simples em~$P$ (acho que eu teria que fazer uma indução em
pares~$\<p,\alpha>$).  Talvez ela tenha que ficar fora mesmo---i.e., na
definição da relação~$\eval{}$, igual o original---ou talvez haja alguma
forma de implementá-la em~$\step{}$ sem complicar as provas.
\end{TODO}

The reaction-step relation ($\step{}$), defined in \Cref{Section:step}
computes a single step within a reaction.  To compute a full reaction we now
define the reaction relation~$\eval{n}\subseteq\Delta\times{P}$ such
that~$\<p,\alpha>\eval{n}{p'}$ iff program~$p$ in initial state~$\alpha$
evaluates (reacts) to a modified program~$p'$ in the empty state in a finite
number of steps.  More formally, for all~$\alpha\in{E}$ and~$p$, $p'\in{P}$:
%%
\begin{equation*}
  \<p,\alpha>\eval{n}{p'}\quadtext{iff}\<p,\alpha>\stepx{n'}\<p,\nil>\,,
\end{equation*}
%%
where~$\stepx{}$ is the reflexive-transitive closure of relation~$\step{}$.
We use~$\<p,\alpha>\eval{n}{p'}$ as an abbreviation
for~$\<p,\alpha,n>\eval{}p'$.

Finally, we describe the complete execution of a \Ceu\ program, given some
sequence of external events, by applying the reaction relation repeatedly to
each event in the sequence, with the reaction number incremented after each
application.


\section{Properties of the reaction relation}
\label{Section:reaction-properties}

\begin{thm}[Determinism]
For all~$p$, $p_1$, $p_2\in{P}$, $\alpha\in{E^*}$, and~$n\in{N}$:
\begin{equation*}
(\<p,\alpha>\eval{n}p_1\quadtext{and}\<p,\alpha>\eval{n}p_2)
\quadtext{implies}p_1=p_2\,.
\end{equation*}
\end{thm}
\begin{proof}
By induction on the number of reaction steps.  \FIXME{\dots}
\end{proof}

\begin{thm}[Termination]
For all~$p$, $p'\in{P}$, $\alpha\in{E^*}$, and~$n\in{N}$:
\begin{equation*}
(\exists!p'\in{P})(\<p,\alpha>\eval{n}p')\,.
\end{equation*}
\end{thm}
\begin{proof}
By induction on the number of reaction steps.  \FIXME{\dots}

\begin{TODO}
  Talvez eu tenha que partir para uma indução em pares
  ordenados~$\<p,\alpha>$, já que tanto~$p$ pode diminur quanto~$\alpha$.
\end{TODO}
\end{proof}


\section{Alternative formulation (Pure \Ceu)}
\label{Section:pure-ceu}

\begin{TODO}
Uma abordagem alternativa seria simplificar a formalização.  Minha proposta
é, num primeiro momento, definir uma versão reduzida (pura) de
\Ceu\ retirando da linguagem as construções \ceu{\Mem(v)} e
\ceu{\Ifelse{v}{p_1}{p_2}} e talvez outras coisas, e.g., loop e finalização.
E dado essa versão pura, tentar definir uma semântica big-step que se
comporte da maneira desejada, i.e., avaliação da esquerda para direita com
instruções \emph{await} atrasadas.  (Eu acho mais fácil trabalhar com a
big-step, mas se não tiver jeito a gente volta para a small-step.)

Num segundo momento, minha idéia é pegar as semânticas construtivas de
Esterel e ver como eles resolvem o problema do loop causal no \emph{await}
que acorda no mesmo ciclo.  Talvez seja interessante não atrasar os
\emph{awaits} em \Ceu.

Feito isso, i.e., definida uma base semântica e provados o determinismo e
convergência, eu posso partir para as provas de equivalência entre programas
e, quem sabe, encontrar possíveis formas normais.

(A propósito, do jeito que está definido hoje as reações de \Ceu\ não
retornam nada para o ambiente, i.e., não há evento externo de saída.  É isso
mesmo?)
\end{TODO}


\appendix
\section{Auxiliary definitions}
\label{Appendix:extra}


\subsection{The clear function}
\label{Appendix:extra:clear}

The clear function~$\clear\colon{P}\to{P'}$ concatenates in a sequence the
body of all active finalization instructions~\ceu{\Fin p'} within some
program~$p$.  Function~$\clear$ is defined by recursion on~$P$ as follows:
\begin{align*}
\clear(\ceu\Skip)&=\ceu\Skip\\
\clear(\ceu{\Mem(v)})&=\ceu\Skip\\
\clear(\ceu{\Await(e)})&=\ceu\Skip\\
\clear(\ceu{\Emit(e)})&=\ceu\Skip\\
\clear(\ceu{\Break})&=\ceu\Skip\\
\clear(\ceu{\Ifelse{v}{p_1}{p_2}})&=\ceu\Skip\\
\clear(\ceu{p_1;p_2})&=\clear(p_1)\\
\clear(\ceu{\Loop p_1})&=\ceu\Skip\\
\clear(\ceu{p_1\And p_2})&=\ceu{\clear(p_1);\clear(p_2)}\\
\clear(\ceu{p_2\Or p_2})&=\ceu{\clear(p_1);\clear(p_2)}\\
\clear(\ceu{\Fin p_1})&=\ceu{p_1}\\
\clear(\ceu{\Awaiting(e,n)})&=\ceu\Skip\\
\clear(\ceu{\Emitting(n)})&=\ceu\Skip\\
\clear(\ceu{p_1\Atloop p_2})&=\ceu\Skip\\
\end{align*}

The form of program~$p'\in{P'}\subseteq{P}$ returned by~$\clear$ is given by
the grammar:
\begin{equation*}
p'\in{P'}\Coloneqq\ceu{\Skip}\mid\ceu{\Mem(v)}\mid\ceu{p_1';p_2'}\,.
\end{equation*}
(Note that the definition of~$\clear$ assumes that the body~$p_1$ of all
instructions~$\ceu{\Fin p_1}$ is necessarily a member of~$P'$, i.e., $p_1$
is either~\ceu\Skip\ or~\ceu{\Mem(v)} or some finite sequence of such
instructions.)


\subsection{The is-blocked predicate}
\label{Appendix:extra:is-blocked}

The is-blocked predicate~$\#\colon{P}\times{E^*}\times{N}\to\{\top,\bot\}$
checks if all execution trails of a given program~$p$ in state~$\alpha$ and
reaction~$n$ are blocked, i.e., if all its trails are hanged in instructions
\ceu\Awaiting\ or \ceu\Emitting, which cannot be consumed in the current
reaction step.  Function~$\#$ is defined by recursion on~$P$ as follows:

\begin{align*}
\#(\ceu\Skip,e\alpha,n)&=\bot\\
\#(\ceu{\Mem(v)},e\alpha,n)&=\bot\\
\#(\ceu{\Await(e')},e\alpha,n)&=\bot\\
\#(\ceu{\Emit(e')},e\alpha,n)&=\bot\\
\#(\ceu{\Break},e\alpha,n)&=\bot\\
\#(\ceu{\Ifelse{v}{p_1}{p_2}},e\alpha,n)&=\bot\\
\#(\ceu{p_1;p_2},e\alpha,n)&=\#(p_1,e\alpha,n)\\
\#(\ceu{\Loop p_1},e\alpha,n)&=\bot\\
\#(\ceu{p_1\And p_2},e\alpha,n)&=\#(p_1,e\alpha,n)\land\#(p_2,e\alpha,n)\\
\#(\ceu{p_1\Or p_2},e\alpha,n)&=\#(p_1,e\alpha,n)\land\#(p_2,e\alpha,n)\\
\#(\ceu{\Fin p_1},e\alpha,n)&=\bot\FIXME{??}\\
\#(\ceu{\Awaiting(e',n')},e\alpha,n)&=e\ne{e'}\lor{n=n'}\\
\#(\ceu{\Emitting(n')},e\alpha,n)&=\|e\alpha|\ne{n'}\\
\#(\ceu{p_1\Atloop p_2},e\alpha,n)&=\#(p_1,e\alpha,n)\FIXME{??}\\
\end{align*}

\end{document}
